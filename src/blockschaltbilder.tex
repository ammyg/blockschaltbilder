\newcommand{\Summationsstelle}[4][]{ %
  \begin{scope}[thick, #1]
    % Runder leerer Knoten mit vordefinierter Größe
    \node[draw, circle, minimum width = #4, minimum height = #4] (#2) at (#3) {};
  \end{scope} %
}

\newcommand{\Verzweigung}[4][]{ %
  \begin{scope}[inner sep = 0 pt, outer sep = 0 pt, thick, #1]
    % Leerer Knoten mit vordefinierter Größe
    \node[circle, minimum width = #4, minimum height = #4] (#2) at (#3) {};
    % Gefüllter Kreis
    \draw[solid, fill] (#2) circle (0.5*#4);
  \end{scope} %
}

\newcommand{\UeFunk}[5][]{ %
  \begin{scope}[thick, #1]
    % Knoten mit vordefinierter Größe
    \node[minimum width = #4, minimum height = #4] (#2) at (#3) {#5};
    % Block
    \draw (#2.north east) rectangle (#2.south west);
  \end{scope} %
}

\newcommand{\PGlied}[5][]{ %
  \begin{scope}[thick, #1]
    % Leerer Knoten mit vordefinierter Größe
    \node[minimum width = #4, minimum height = #4] (#2) at (#3) {};
    % Beschriftung über dem Block
    \node[fill = none, above, align = left, text width = #4] at (#2.north) {#5};
    % Block
    \draw (#2.north east) rectangle (#2.south west);
    % Horizontale Linie
    \draw ([yshift = -0.225*#4] #2.north east) -- ([yshift = -0.225*#4] #2.north west);
  \end{scope} %
}

\newcommand{\IGlied}[5][]{ %
  \begin{scope}[thick, #1]
    % Leerer Knoten mit vordefinierter Größe
    \node[minimum width = #4, minimum height = #4] (#2) at (#3) {};
    % Beschriftung über dem Block
    \node[fill = none, above, align = left, text width = #4] at (#2.north) {#5};
    % Block
    \draw (#2.north east) rectangle (#2.south west);
    % Diagonale Linie
    \draw (#2.north east) -- (#2.south west);
  \end{scope} %
}

\newcommand{\DGlied}[5][]{ %
  \begin{scope}[thick, #1]
    % Leerer Knoten mit vordefinierter Größe
    \node[minimum width = #4, minimum height = #4] (#2) at (#3) {};
    % Beschriftung über dem Block
    \node[fill = none, above, align = left, text width = #4] at (#2.north) {#5};
    % Block
    \draw (#2.north east) rectangle (#2.south west);
    % D-Eck
    \draw ([xshift = 0.300*#4] #2.north west) |- ([yshift = 0.225*#4] #2.south east);
  \end{scope} %
}

\newcommand{\TZGlied}[6][]{ %
  \begin{scope}[thick, #1]
    % Leerer Knoten mit vordefinierter Größe
    \node[minimum width = #4, minimum height = #4] (#2) at (#3) {};
    % Beschriftung über dem Block
    % \mbox stellt sicher, dass \hfill auch bei fehlenden Argumenten aktiv ist
    \node[fill = none, above, align = center, text width = #4] at (#2.north) {\mbox{#5} \hfill \mbox{#6}};
    % Block
    \draw (#2.north east) rectangle (#2.south west);
    % Totzeit-Eck
    \draw ([xshift = 0.400*#4] #2.south west) |- ([yshift = -0.225*#4] #2.north east);
  \end{scope} %
}

\newcommand{\PTEinsGlied}[6][]{ %
  \begin{scope}[thick, #1]
    % Leerer Knoten mit vordefinierter Größe
    \node[minimum width = #4, minimum height = #4] (#2) at (#3) {};
    % Beschriftung über dem Block
    % \mbox stellt sicher, dass \hfill auch bei fehlenden Argumenten aktiv ist
    \node[fill = none, above, align = center, text width = #4] at (#2.north) {\mbox{#5} \hfill \mbox{#6}};
    % Block
    \draw (#2.north east) rectangle (#2.south west);
    % PT1-Kurve
    \begin{scope}[shift={(#2.south west)}]
      \draw %
        (0.00, 0.00) .. controls (0.10*#4, 0.80*#4) and (0.40*#4, 0.80*#4) .. %
        ([yshift = -0.18*#4] #2.north east);
    \end{scope} %
  \end{scope} %
}

\newcommand{\PTZweiGlied}[6][]{ %
  \begin{scope}[thick, #1]
    % Leerer Knoten mit vordefinierter Größe
    \node[minimum width = #4, minimum height = #4] (#2) at (#3) {};
    % Beschriftung über dem Block
    % \mbox stellt sicher, dass \hfill auch bei fehlenden Argumenten aktiv ist
    \node[fill = none, above, align = center, text width = #4] at (#2.north) {\mbox{#5} \hfill \mbox{#6}};
    % Block
    \draw (#2.north east) rectangle (#2.south west);
    % PT2-Kurve
    \begin{scope}[shift={(#2.south west)}]
      \draw %
        (0.00,    0.00   ) .. controls (0.07*#4, 0.00   ) and (0.10*#4, 0.10*#4) .. %
        (0.18*#4, 0.40*#4) .. controls (0.26*#4, 0.70*#4) and (0.35*#4, 0.85*#4) .. %
        (0.43*#4, 0.85*#4) .. controls (0.49*#4, 0.85*#4) and (0.58*#4, 0.75*#4) .. %
        (0.64*#4, 0.69*#4) .. controls (0.70*#4, 0.63*#4) and (0.76*#4, 0.59*#4) .. %
        (0.82*#4, 0.59*#4) .. controls (0.88*#4, 0.59*#4) and (0.97*#4, 0.60*#4) .. %
        ([yshift = -0.35*#4)] #2.north east);
    \end{scope} %
  \end{scope} %
}

\newcommand{\NeueEA}[5]{ %
  % Füge neue Koordinaten hinzu:
  % Nordseite (oben), von links nach rechts
  \foreach \i in {1, 2, ..., #2}
  {
    \coordinate (#1--north \i) at
      ($(#1.north west)!{\i/(#2 + 1)}!(#1.north east)$);
  }
  % Ostseite (rechts), von oben nach unten
  \foreach \i in {1, 2, ..., #3}
  {
    \coordinate (#1--east \i) at
      ($(#1.north east)!{\i/(#3 + 1)}!(#1.south east)$);
  }
  % Südseite (unten), von links nach rechts
  \foreach \i in {1, 2, ..., #4}
  {
    \coordinate (#1--south \i) at
      ($(#1.south west)!{\i/(#4 + 1)}!(#1.south east)$);
  }
  % Westseite (links), von oben nach unten
  \foreach \i in {1, 2, ..., #5}
  {
    \coordinate (#1--west \i) at
      ($(#1.north west)!{\i/(#5 + 1)}!(#1.south west)$);
  }
}
